\section {Introduction}
These standards and guidelines have been chosen to promote a consistent
style, clarity and organization in AIPS++ code.  {\em Standards} are
coding practices which everyone must follow; {\em guidelines} are
recommended practices, and are identified below by ($\star$).  

{\it Effective C++} by Scott Meyers is a good guide to C++ programming.  
Most of his guidelines apply to AIPS++. 
%%---------------------------------------------------------------------------
\section {Code Organization}
%%---------------------------------------------------------------------------
\subsection {Classes}  Every class will have a header file,\footnote{See
{\em code/install/codedevl/template-class-h}} 
an implementation file,\footnote {Neither totally inlined classes, nor
abstract base classes need an implementation file.} a test program
and (optionally) a demo program.  Very closely related
classes\footnote {A good example is a read-only class, and a read-write
class derived from it.}  may sometimes be combined into single header
and implementation files.\footnote {If you do this, make sure that the
documentation sections -- {\em synopsis, etymology, motivation}, etc. -- 
are completed for {\em every} class contained in the file.}

The test program(s) shall exercise every member function, every exception, and 
cover at least 90\% of the lines of code in the implementation file.\footnote
{Special language tools can measure coverage.  See AIPS++ Note 170,
``The AIPS++ Code Review Process'' for suggestions on writing test programs.}
%%---------------------------------------------------------------------------
\subsection {Modules} Closely related classes should be organized into a module
-- which requires an appropriately named, separate directory.\footnote
{The Tables module, {\em code/aips/implement/Tables} is a good example to 
study.}  The module will have a ``module header file'' -- whose primary 
purpose is to simplify things for the application programmer.  Whenever 
possible, application programmers should not be required to learn the
intricacies of a complicated module, nor to understand specific
implementation classes within the module.  They should only have to
include a single header file into their application program for each
module they wish to use; to study the documentation contained (as
comments) in that header file; and to start writing code.  A library
(or infrastructure) programmer, however, will almost certainly skip
the module header, and include only those class header files from the
module directory which they need.  This discriminating approach ensures
that dependency-driven recompilation is kept to a minimum.

The module header file\footnote{See {\em
code/install/codedevl/template-module-h}} will live in the module's parent
directory: {\em code/aips/implement/Tables.h}, for example, is the module
header file for, and describes the classes in {\em code/aips/implement/Tables/}.
%%---------------------------------------------------------------------------
\subsection {Global Functions} It is sometimes appropriate to write
global functions to act on class objects, rather than creating
class member functions.  As a general rule, you should do this if
there is no class object state preserved from one function call to the next,
and all of the information about the class object/s can be obtained
from the public interface.  There are two places where global functions
may be declared and defined:  as part of the {\em .cc} and {\em .h} files
of the class they are most closely associated with, or in their own, separate 
{\em .cc} and {\em .h} files.  Use these criteria to decide:
\begin{enumerate}
\item If there is a large conceptual distance between the functions and
the class, then it is probably best to use separate files.
\item If there are a large number of related global functions, use
separate files.
\item If the functions are templated, using separate files will reduce
template instantiation dependencies.
\end{enumerate}
%%---------------------------------------------------------------------------
\subsection {File Size ($\star$)}
Files should rarely -- if ever -- exceed 2000 lines:  a 1000 line
maximum will usually apply.  Well-designed classes often have short header
and implementation files, usually less than 500 lines including documentation.
%%---------------------------------------------------------------------------
\subsection {Function Length ($\star$)}
Functions should rarely exceed 100 lines in length; shorter, well-focused
functions should dominate, and will usually be 50 lines or fewer.
%%---------------------------------------------------------------------------
\section {Documentation and Naming}
\subsection {Documentation in Header Files}
Header files shall contain clear and complete documentation for classes and
modules.  Templates (or ``boilerplate'') for these files are 
{\em code/install/codedevl/template-class-h} and {\em template-module-h}.
That directory also has templates for other standard kinds of files.
%%---------------------------------------------------------------------------
\subsection {Documentation in Implementation Files}
Background, usage, and overall design is presented in class header files,
and implementation files do not need to repeat them.  But non-obvious
sections of code do need to be accompanied by explanation, including 
references to manuals or texts as appropriate. ``Obvious'' is to be
judged from the perspective of a competent programmer, generally 
knowledgeable -- but not necessarily expert -- in the program domain.  The
guiding principle is:  do not force those who read your code to spend 
unnecessary time deciphering it.  It will {\em always} be a net savings
of time if you document what you have done, and why you did it.
%%---------------------------------------------------------------------------
\subsection {Names for Classes, Functions and Variables}
All class, function and variable names will reflect a balance between
clarity and convenience, with clarity having priorty.  Idiomatic 
abbreviations, acronyms and contractions are discouraged.
Names for classes, structures, and enumerated types shall begin with
an uppercase letter.  Identifers which consist of more than one word
(a recommended practice) shall use uppercase at the beginning of each
new word (``{\em GoodClassName}'').  Local variables, class data members,
class member functions, and global functions names shall begin with
lowercase letters (``{\em goodVarName, goodFunctionName}'').  

Glish function names follow a different policy: they shall use lower
case letters exclusively, and the underscore character shall be used,
where necessary, to increase readability.  For example: ``{\em
good\_function\_name}''\footnote {The aips++ end-user will frequently
type Glish function names; this policy is for their typing convenience.}.
%%---------------------------------------------------------------------------
\subsection {Names for Files} The only restrictions here are
commonsense, and the Unix library archive utility ``{\em ar}''.  The
first suggests that file names should be long enough to convey meaning
to the reader, but not so long as to be a burden to type. This
translates to: file names should be from about 8 to about 30
characters long.

``{\em ar}'' requires that all object file names be unique in the first 14
characters.  

Underscores are not allowed in file names.

Any file (a shell script, or a C++ program, for example) which can be
invoked at the command line as an executable program, must be named
with lower case chararacters only.  (This policy follows the glish function
naming policy of the previous section; both are designed to present the 
end user with only lower case commands to type, but note that underscores
are prohibited in file names.)

Use the standard extensions ({\em .c, .cc, .h, .f, .g}).
%%---------------------------------------------------------------------------
\subsection {Format for Dates}
Programmers (and code reviewers) must enter dates when documenting code.
These dates indicate, for example, when code has been reviewed, and when a list 
of ``to do'' tasks was last updated.\footnote{See, for instance, the 
documentation tags {\em reviewed} and {\em todo} in {\em template-class-h}.}
AIPS++ uses a single mandatory format for expressing dates.  It has the virtue
of being unambiguous, it is comprehensible to readers around the world, and
it is also used by the project's version control system {\em RCS}.  This 
format is {\em yyyy/mm/dd}. An example is {\em 1995/02/27}.
%%===========================================================================
\section {Coding}
\subsection {Forward Declarations ($\star$)}
{\em \#include} only those header files your class absolutely requires.  Use
forward declarations when that will suffice.\footnote {This may be difficult
to determine, due to the complexities of template instantiation.}
%%---------------------------------------------------------------------------
\subsection {Protected Data Members ($\star$)}
Avoid protected data memebers, using protected member functions for access
to private data instead.
%%---------------------------------------------------------------------------
\subsection {Access to Private Data}
Do not return pointers or references to private data, except to accomodate
Fortran or C libraries, of if efficience absolutely requires it.
%%---------------------------------------------------------------------------
\subsection {Label All Virtual Functions Explicitly}
Virtual functions in a derived class shall be explicitly labelled with
the {\em virtual} keyword in the class declaration, even though this is
 not required by the language.
%%---------------------------------------------------------------------------
\subsection {Document Loose Ends}
Indicate unfinished or questionable code with the documentation extractor's
tag {\em $<$todo$>$}.
%%---------------------------------------------------------------------------
\subsection {Standard Class Member Functions}
Provide definitions for all of the standard member functions which the
compiler would (otherwise) generate automatically:  default constructor,
destructor, coy constructor, and the assignment operator.  For any of 
these which you wish to disallow, declare them private, and create
minimal implementations.  For example:
\begin{verbatim}
   ...
   private:
      SomeClass () {;}   // disable the default constructor
   ...
\end{verbatim}
%%---------------------------------------------------------------------------
\subsection {The Order of Function Arguments ($\star$)}
Functions shall be declared so that their arguments (parameters) appear in 
this order: output parameters first; input/output parameters next; 
input parameters last.\footnote {This order may seem unnatural, but it is a 
direct consequence of the C++ rule that default parameters appear
{\em last} in the parameter list.}  Functions which return only one value 
should ususally use the function's return-value to return that value, rather
than add an additional output parameter to the parameter list.  If
a function modifies its single argument {\em in place}, then that
argument is best understood as an input/output argument.\footnote
{But also consider {\em strcpy} from the 
standard {\em C} library, which both modifies an argument, and returns 
that same value as the function return value.}
%%---------------------------------------------------------------------------
\subsection {Formal Arguments and Default Parameters}
Specify formal arguments and default parameters in the function declaration;
use the same names in the function definition.
%%---------------------------------------------------------------------------
\subsection {Return Types}
Specify explicit return types for all functions.  Return an appropriate
integer value from {\em main}.
%%---------------------------------------------------------------------------
\subsection {Exceptions}
Exceptions should be reserved for truly exceptional circumstances, and
not used to replace status flags or condition tests.
%%---------------------------------------------------------------------------
\subsection {Compiler Warnings}
All code should compile without producing warnings from the 
compiler.  Unfortunately, some compilers produce spurious warnings, especially
if the highest warning level is used.  All warnings should be 
examined, but those which are spurious may be ignored.
%%---------------------------------------------------------------------------

