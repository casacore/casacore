
\section{Background}
VLA Observe is an important legacy-program for the NRAO.  It provides most VLA
users with the only practical way to schedule their VLA observations.  By most 
accounts it does a reasonable job of scheduling the VLA.  
The current version of Observe was conceived in 1988 and completed in March of
1991. Observe has been fairly stable since 1991.  A number of bug fixes and
small enhancements have been made (on the order of 30 minor revisions have been
recorded).  The program was ported to C++ in the summer of 1996 and no
minor releases have been made since its official release on Dec 31, 1996.
It has approximately 44000 lines of code
including comments (comments are fairly sparse).  Over the years several
problems have surfaced with Observe:
\begin{enumerate}
\item It is key-context
driven with a "custom" windowing scheme (rather than event driven),
\item  Navigation is tricky for novice and occasional users,
\item User documentation is out-of-date, 
\item There is no design/implementation documentation
\item Newer features are poorly documented,
\item It is difficult to add new features, 
\item Maintenance is done by one person with no backup, and
\item Most common, \textbf{keypad problems!}
\end{enumerate}
In addition to the above problems, when Observe was
originally written most users had terminals connected to a central computer.
The general computing environment at NRAO (and for NRAO's users) has changed.
We now
\begin{itemize}
\item have PC/Workstations,
\item use GUI based applications,
\item have AIPS++,
\item use the World Wide Web, and
\item have Java available.
\end{itemize}

\section{Future Direction of Observe}
\subsection{Goals \& Options}

The changed nature of computing since Observe's introduction have caused us
to review Observe's current status and reflect on how we might "modernize" it.
Modernization might include:
\begin{itemize}
\item   Remove the keypad dependence,
\item   Improve the user interface,
\item   Improve portability,
\item   Take advantage of AIPS++ Measures (coordinates, time, Doppler
tracking),
\item   Easier maintenance, 
\item   Easier modification, and 
\item   Better access to documentation.
\end{itemize}

There are two options:
\begin{enumerate}
\item Revise user and programmer documentation.

\item Change to an event driven based program.

\end{enumerate}


\subsection{Discussion}

Option 1 leaves things much as they are.  Current documentation would be
updated and there would be some programmer documentation.  This option doesn't
improve the interface for the observer or improve the maintainability of
Observe.  Screen real-estate in the current version is very limited, it is
difficult to add new features.  Information is hidden and navigation is not
always straight forward.


Option 2 requires rewriting and to some extent redesigning Observe.
A redesign would allow a more natural interface (and avoid the keypad
altogether).  The current version closely couples the user interface with 
program functionality. A redesign would allow loosening the coupling, so
the user interface and scheduling/file-checking functionality are more
logically separated.

There are several choices for an event-driven program:
\begin{enumerate}
\item "Classic X11",
\item Tcl/Tk, or
\item GlishTk.
\item Java,
\end{enumerate}

In all choices, \textbf{almost none} of the existing code will be preserved. 
Translating many of
the concepts and algorithms from VLA Observe to the new version should be
fairly straight forward..
The AIPS++ Measures code will be used for Doppler corrections, coordinate and
time conversions (probably as a server object).
All AIPS++ tasks/objects use the Measures code, so Observe's numbers will be
consistent with those of AIPS++.

A "Classic X11" based program would require compiling and linking for each
platform. 
This is very much a classical approach for providing software.  Users down-load
Observe and ancillary files to their local machines, configure for their local
environment,  and run Observe
on the local machine.
Often we see files submitted using an out-of-date version of Observe.
Porting to a non-supported architecture while not difficult does require time
and some effort.  Often when a port is needed, a computer of that architecture
is not available locally.

Tcl/Tk and GlishTk are a Tk-widget-based X11-toolkit called from a
command interpreter.  
Tcl/Tk uses the wish shell and GlishTk is the AIPS++ command
interpreter.  Writing Observe in GlishTk provides a straight-forward interface
with existing AIPS++ code. While being quick-to-prototype,
both have their drawbacks.
Some problems with Tcl/Tk include:
\begin{itemize}
\item AIPS++ measures code may not run on all platforms,
\item The code would have to be distributed (latest version not always
guaranteed),
\item Requires the users to have the proper Tcl/Tk configuration, and
\item It can be difficult to debug Tcl/Tk code.
\end{itemize} 

Problems with GlishTk include:
\begin{itemize}
\item Requires AIPS++ to be installed,
\item Current version GlishTk doesn't support all of Tk features/options,
\item It can be difficult to debug GlishTk code, and
\item Latest version not always guaranteed.
\end{itemize}

A Java based program offers the hope of "write-once run-anywhere" (no
porting issues).  A user
connected to the net could be guaranteed the latest version (including the
latest calibrator positions and NRAO defaults). 
Observe seems particularly well suited to "net".  Links are possible between
Observe and
the most-current (best?) advice on observing strategies, interference problems,
and other timely observing information.  User could also down-load the Java
byte-code and run Observe locally.
The biggest drawback to Java is, "it is cutting edge".

\subsection{Redesign/Reimplementation}

There are three phases in redesigning and reimplementing Observe:
\begin{description}
\item[First,] Modify the design to remove the interface from the
           scheduling/verification routines;
\item[Second,] Code/Test the foundation; and
\item[Third,] Code/Test the user interface.
\end{description}

Considering the nature of observe, the language of choice is Java.  Why Java?
\begin{itemize}
\item Portability.  Java will run on most platforms as is, i.e. no
recompilation/relinking.
\item Can be run effectively over the network.
\item Provides the most seamless distribution. A VLA observer only has to
connect to the observe web page to run the latest version.
\end{itemize}
The last point provides the most compelling reason.

The following list is a "back-of-the-envelope" targets
list that would be refined after the design work.  Time estimates indicate
the relative amount of work (rather than the actual time).  The time required
to implement the GUI is likely to be independent of implementation (Java, X11,
Tcl/Tk, or GlishTk).

Testing individual components and programs will follow the AIPS++
model.  As we finish parts of the project, we prepare test-cases with
known results. At
regular intervals the test-cases are run and compared with expected results.
If a test fails we can track it down before it becomes a bug report.
Also, we can run the test-cases after any changes to ensure no unforeseen
problems have occurred.  Testing the GUI is more labor intensive and will
require interactive testing by NRAO staff members.


\begin{longtable}{l}
VLA Observe -- Targets\\
Phase 1 -- Design - deliverable (2 weeks)\\
\hline\\
\hspace{0.25in} rudimentary design of current Observe\\
\hspace{0.25in} redesign Observe\\
Phase 2 -- Foundation (17 weeks)\\
\hline\\
ASCII Text to Observe file - deliverable (9 weeks)\\
   \hspace{0.25in}conversion into Java\\
   \hspace{0.25in}parser\\
   \hspace{0.5in}   user defined\\
   \hspace{0.5in}   observe file\\
   \hspace{0.25in}interface with AIPS++ Measures\\
   \hspace{0.25in}scheduling routines\\
   \hspace{0.25in}frequency checking\\
   \hspace{0.25in}output routines\\
   \hspace{0.25in}unit testing
\\
Reports - deliverables(4 weeks)\\
   \hspace{0.25in}analysts\\
   \hspace{0.25in}scheduling\\
   \hspace{0.25in}frequency\\
   \hspace{0.25in}time-on-source\\
   \hspace{0.25in}summary\\
   \hspace{0.25in}unit testing
\\
Databases - NRAO/user (3 weeks)\\
   \hspace{0.25in}frequency setups - deliverable\\
   \hspace{0.25in}calibrator - deliverable\\
   \hspace{0.25in}source - deliverable\\
   \hspace{0.5in}  position\\
   \hspace{0.5in}  orbital elements\\
   \hspace{0.5in}  interpolated from table\\
   \\
Integration into a scheduler - deliverable(1 week)\\
\\
\\
Phase 3 -- Interactive Interface (20 weeks)\\
\hline\\
Scheduler GUI (10 weeks)\\
   \hspace{0.25in}screen layouts - deliverable\\
   \hspace{0.25in}source catalog - deliverable\\
   \hspace{0.25in}frequency/correlator setups - deliverable\\
   \hspace{0.25in}scan editor - deliverable\\
   \hspace{0.25in}visual loser - deliverable\\
   \hspace{0.5in}   LO calculator\\
   \hspace{0.5in}   band/interference plots\\
   \hspace{0.5in}   Doppler tracking\\
   \hspace{0.25in}observe file viewer - deliverable\\
   \hspace{0.25in}scan contraction/expansion\\
   \hspace{0.25in}starting conditions - deliverable\\
   \hspace{0.25in}editing schemes\\
\\
User Testing (4 weeks)\\
\\
Planning Tools - deliverables (2 weeks)\\
   \hspace{0.25in}uptime \\
   \hspace{0.25in}uv-coverage\\
   \hspace{0.25in}az-el plots\\
   \hspace{0.25in}calibrator flux history\\
\\
Distribution - deliverable (1 week)\\
\\
Documentation (4 week)\\
   \hspace{0.25in}programmer - deliverable\\
   \hspace{0.25in}user - deliverable\\
   \hspace{0.5in}   tutorial\\
   \hspace{0.5in}   on-line help\\
\\
Phase 4 -- Futures (?)\\
\hline\\
   \hspace{0.25in}suggested schedule\\
     \hspace{0.5in}time range, source, frequency, pick calibrators, etc...\\
   \hspace{0.25in}optimal scheduling\\
   \hspace{0.25in}incorporate VLA plan\\
\end{longtable}
