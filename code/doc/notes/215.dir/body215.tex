\newcommand{\aips}{\textit{AIPS++}}
\section{Overview}
The \aips documentation is a combination of standards and tools.
The standards are HTML, LaTeX, and \htmladdnormallink{\aips guidelines}{}. 
The tools are cxx2html, help2tex, and
latex2html, with a dash of shell programming thrown in.
All \aips documentation is available via a web interface.  
Most of the user and system
documentation and \aips Notes and Memos are written using the LaTeX
mark up language. Some older documents may be in texinfo which was abandoned
in favor of LaTeX. C++ classes are documented in header files using 
special cxx2html tags.

Cxx2html, help2tex, and latex2html are all perl scripts. 
A working knowledge of perl is helpful to understand them and essential for 
debugging.

All documentation source is found in the "code" tree.  The make system's docsys
target generates and moves postscript/html files into the "docs" tree.

\section{LaTeX based documents}
\subsection{General Documentation}
\subsubsection{General Notes}
Documents in the \aips Notes and Memo series are mostly written using LaTeX.
All new documents should be written in LaTeX.
There are a few notes that are plain ASCII (.txt) files.  The make system
generates postscript
files from files with .latex and .tex extensions.  Makefile.doc located in
code/install contains all the rules for generating postscript and html files
and moving them into the "docs" tree.

The \aips Notes and Memos series each contain a latex file that has the titles
and authors for each \htmladdnormallink{Note}{../notes/notes.html} and 
\htmladdnormallink{Memo}{../../memos/memos/memos.html}.  Each time a note or memo
is checked in, the index.tex file in code/doc/memos/memos.dir and
code/doc/notes/notes.dir needs to be edited to include the new note or memo.

\subsubsection{Generating HTML}
HTML documents are not automatically generated from LaTeX source.  The author
needs to add a line to the directory's makefile similar to:
\begin{verbatim}
documentroot := -split +2
\end{verbatim}

where documentroot is the file's name less the .latex extension.  After
specifying this line 
latex2html will be run on that document during a sneeze.  Most \aips documents
should use
\begin{verbatim}
-split +1 or +2.
\end{verbatim}


The default behavior of latex2html may altered using at .latex2html-init
file.  Typically the default split level, and appearance of the HTML pager
header and footer will be customized.

Latex2html uses absolute path names to its icons, the shell script
redoicons.sh (code/install/docutils) turns the absolute paths into relative
ones.  Using relative icons avoids have latex2html generated HTML pages
use NRAO as an icon server. To get relative icons sites should specify two
variables in thier makedefs file ICONSERVER and ICONS2LOCAL.  ICONSERVER is
the ICONSERVER variable specified in the latex2html script.  At the AOC,
we have the following two lines in our makedefs to generate local icons:
\begin{verbatim}
ICONSERVER := http://www.nrao.edu/icons/latex2html
ICONS2LOCAL := $(AIPSARCH)/bin/redoicons.sh
\end{verbatim}

For more details about the html commands available from latex2html please 
consult the
 \htmladdnormallink{latex2html documentation}{ 
http://www-dsed.llnl.gov/files/programs/unix/latex2html/manual/manual.html}
found at\\ 
\htmladdnormallink{http://www-dsed.llnl.gov/files/programs/unix/latex2html/manual/manual.html}{http://www-dsed.llnl.gov/files/programs/unix/latex2html/manual/manual.html}.

Latex2html should be found in the system area (/opt/local/bin/latex2html at the
AOC).  \aips  uses version 98.2beta5 (1998-Jul-28).  Latex2html requires several utilities
\begin{itemize}
\item Perl (5.003 or later).
\item dvips (5.516 or later, preferably 5.62 or later)
\item gs (4.03 or later).
\item netpbm (1 March 1994 version)
\end{itemize}

Please consult the README distributed with latex2html for more details.

\subsubsection{HTML Links to Other Documents}
There are several methods for producing links to other documents.  To
reference an HTML document, use:\\
\verb!\htmladdnormallink{text}{link to html page}!.\\
Provided an author uses
\verb!\label! commands you may reference other 
LaTeX based document using 
\verb!\externallabels! command and \verb!\htmlref!.

To make a link to the \htmlref{copy}{catalog:catalog.copy.function} command of the
catalog object in the catalog module use \verb!\htmlref!
\begin{verbatim}
\htmlref{copy}{catalog:catalog.copy.function}
\end{verbatim}

The \verb!\externallabels{URL to document}{path to labels file}! command
needs to appear at the top of the document before the \verb!\begin{document}!. 
The URL and path to labels file need to be relative paths.  So for Note 215
\begin{verbatim}
\externallabels{../../user/Refman}{../../user/Refman/labels.pl}
\end{verbatim}

Help files auto-generate some labels based on the following scheme:
\begin{itemize}
\item module
\item module:function
\item module:tool
\item module:object.method
\end{itemize}
There are also two special case where a function and constructor have the
same name:
\begin{itemize}
\item module:tool.constructor
\item module:tool.function
\end{itemize}
Note the constructor and function are added to the end of the label, i.e.
images:image.calc.constructor or images:image.calc.function to identify them
uniquely.
Additional labels may be put in the .help file by the author using the 
\verb!\label! command.

The order of when documents are run through latex2html is important
for external cross referencing. All released 
documentation needs two passes through latex2html to resolve all 
external references (insures the use of the proper labels.pl file).  

Funny things can happen if labels are multiply defined.


Cross referencing into HTML files created by cxx2html is possible using
\verb!\htmladdnormallink!.  To make a reference to the array class
\htmladdnormallink{array class}{../../aips/implement/Arrays/Array.html#Array}
from a note or memo use:
\begin{verbatim}
\end{verbatim}

Please consult the \htmladdnormallink{cxx2html documentation}
{../../html/cxx2html.html} to see how anchors are created.

Cross referencing from an HTML document to something in the User's
Reference Manual or another LaTeX based document is possible.
The link must be identified 
in the generated HTML.  This is not recommended since latex2html generated
file names and anchors will likely change.

A word of caution, relative links in the docs tree may be different than those
found in a programmer's code tree.  Within the code/doc tree relative links
should be OK (i.e. builds of LaTeX documents should cross reference properly
amongst themselves for programmer and system builds).  Relative links to html
pages generated by cxx2html will not be the same for programmer and system
builds!  The implement directory is missing in the docs tree.

\subsection{Adding Notes and Memos}\label{215:addnotes}
\textit{Contributed by Wim Brouw, ATNF}\newline
It took me a while to insert a Note in the system and considering the
difficulty others have had, I make the following points:

\begin{enumerate}
\item Select a note number (I just take the next number available, but maybe
   there is somebody to ask?) - say nnn (e.g. 225)
\item Create (check in) directory code/doc/notes/nnn.dir
\textit{(If it's there then others will know the number is taken.)}
\item Put the body of your latex Note (without embracing document information
   like documentstyle, title, \verb+\begin{document}+ etc) in nnn.dir (e.g.
   ai -l mynote.latex)
\item Create, (in code/doc/notes itself) a file called nnn.latex containing the
   document information, and which includes mynote.latex (without a
   directory!)
   Example:
\begin{verbatim}
\documentclass[11pt]{article}
\usepackage{html, epsf}
%%-----------------------------------------------------------------------------

\begin{document}

\title{NOTE 224 -- AIPS++ Least Squares background}
  

\author{Wim Brouw}

\date{22 January 1999}

\maketitle
%%---------------------------------------------------------------------------
\begin{htmlonly}
\htmladdnormallink{A postscript version of this note is available 
(124kB).}{../224.ps}
\end{htmlonly}

\tableofcontents
      
\input{lsq.latex}
\end{document}
\end{verbatim}

\item Add your note number to the code/doc/notes/makefile (just do like the
   other notes, e.g.:
\begin{verbatim}
224 := -split 0
\end{verbatim}

\item Test if all ok, by making a Postscript version:
   gmake nnn.ps\newline
   And an html version:\newline
   gmake nnn\newline

\item If all is ok, then add your Note to the Notes index:
\begin{verbatim}
   code/doc/notes/notes.dir/nindex.tex
   E.g:
\item[224]
\htmladdnormallink{\textit{AIPS++ Least Squares background}}{../224/224.html}
\linebreak  1999/01/25 Brouw
\end{verbatim}

\end{enumerate}


\textit{Note: Other notes, AIPS++ documents can include your note or parts of your note
by adding the following line to the makefile}
\begin{verbatim}
TIROOT := $(word 1, $(AIPSPATH))/code/doc
EXTRA_TEXINPUTS := $(TIROOT)/memos/111.dir:$(TIROOT)/notes/156.dir
/notes/196.dir
\end{verbatim}
\textit{you then use the} \verb+\input{file}+
\textit{ to include it the other document.
The HowTos (code/doc/reference/HowTos.latex) is a good example.}


\subsection{Help System}
The help system is a special type of LaTeX document.  User documentation for
packages, modules, objects, and functions are written in .help files.
Authors use the \htmlref{aips2help.sty}{197:aips2help} package to format their
help text.  The perl script \htmlref{help2tex}{system:help2tex} is run on the
.help file to produce "standard" LaTeX so no special modules are needed by
latex2html.
Help2tex provides a generate consistent looking documentation if the
aips2help.sty commands and environments are used.

\subsubsection{How it works}
The .help files are usually found in same directory as the "code".
The docsys target for code/doc/Refman has been enhanced to do the following:
\begin{enumerate}
\item Collect all the .help files into a temporary directory.
\item run help2tex (a perl script, code/install/docutils) on each package in
the system. Help2tex is run run twice on each package first to to generate
a .htex file (found in docs/user/helpfiles) and then to generate an atoms.g
file used for glish command-line help.
\item run latex to generate the postscript file in docs/user/Refman.ps.
\item run latex2html to generate the html files in docs/user/Refman.
\end{enumerate}

\subsubsection{Adding New Packages}
The reference manual has been broken into several books due to pool space limitation of
LaTeX.  Creating a new reference book requires several steps.
\begin{enumerate}
\item Create the top-level Package.latex file similar to General.latex. You will need to either
include the package.help file in this new Package.latex file or write some text.
\item edit the makefile in code/doc/user
\begin{enumerate}
 \item add the new package to the HELPPACKS variable
\item define a PACKAGE.s2ps variable (will create lots of ps.gz files for package, module,
and methods)
\item set the split level of Package.latex file (should be := -split +4 -link 2 -short\_index)
\item edit Refman.latex to add the new package to the top-level user reference manual.
\item if needed add the \externallabels commands to other .latex files in the system (only
necessary if you wish to cross reference from a .latex document to the User Reference Manual).
\end{enumerate}
\end{enumerate}



\section{HTML based documents}
\subsection{Top-level WWW pages}
Several WWW pages are found in code/doc.  These pages are "straight" HTML
and copied directly into the docs tree.  Each \aips web page has a standard header
which contains a navigation bar for moving around the documentation tree. There are hooks in
for javascript menus but these have proven to be problematical (i.e. unrepeatable problems with
page display)

Here are the top-level \aips web pages relative code/doc:

\begin{description}
\item[aips++.html] \aips top-level documentation page.
\item[gettingstarted.html]how to get \aips and use it.
\item[user/documentation.html]user documentation material.
\item[glish/glish.html]glish specific material.
\item[learnmore.html]where to get more information about \aips.
\item[programmer/programmer.html]programmer and system documentation.
\item[contactus/contactus.html]defect status, email reflectors, personnel, and mirrors.
\item[newsletters/index.html]latest news from \aips.
\item[faq/faq.html]frequentyl asked questions about \aips.
\item[search/search.html] Search pages for the documentation.
\end{description}

Specific informational HTML pages hang off of these control pages.  Please try and keep the
web tree shallow to avoid burying it.


\subsection{Cxx2html generated documents}
C++ class documentation is contained in the C++ header files. \aips uses 
the perl script
\htmladdnormallink{cxx2html}{../../html/cxx2html.html} (code/install/docutils) to
extract and assemble the class documentation.

\section{Searching \aips Documentation}
\aips uses the htdig system \htmladdnormallink{(http://htdig.sdsu.edu)}{http://htdig.sdsu.edu} to perform searches.
Htdig traverses the web from a specified starting point (\aips home
page) and creates a database of words and documents.  A daily cron job
(/aips++/local/htdig/bin/rundig) is run
to recreate the databases.  The configuration file
(/aips++/local/htdig/conf/htdig.conf)
offers many ways to tune the database.  The search page supplied with
htdig has been customized for \aips and checked into the system
(code/doc/html/aips2search.html). 

Root privileges are needed to install/modify htsearch in the httpd/cgi-bin
directory and add the htdig icons (/tarzan/httpd/icons).

All document searching is currently done on tarzan.aoc.nrao.edu.  Distributing
the search database with the system is not contemplated until there is a
demand for searching local documentation.

\section{\aips Problem Reporting System}
The \aips problem reporting system uses ClearDDTS as the underlying
system for managing and tracking problems (This replaces the older GNATs system). There are
three ways to submit
"a defect":
\begin{enumerate}
\item using \htmladdnormallink{ClearDDTS}{http://aips2.nrao.edu/ddts/ddts_main}, or
\item using the i\htmladdnormallink{AIPS++ bug web page}{../../contactus/reportabug.html}, or
\item bug() (from glish).
\end{enumerate} 
Extensive hard-copy documentation is available for ClearDDTs.


\section{Mailing list setup}
From time-to-time new mailing lists are need.  The process is as follows
\begin{enumerate}

\item Check-in the new mailing list in code/admin/system.
\item Supply the mail alias to NRAO-AOC \htmladdnormallink{helpdesk}{mailto:helpdesk@aoc.nrao.edu}), they put the alias into the master
alias table.  \aips
mailing lists have the form aips2-mailinglist, where mailinglist is the name
of the mailing list.
\item On tarzan as aips2mgr,
edit the /etc/mail/tarzan.aliases-tail file and put in three lines like
\begin{verbatim}
#A comment about the new mailinglist
aips2-newlist: :include:/export/aips++/master/etc/aips2-newlist.list
aips2-newlist-log: "| /export/aips++/master/etc/aipsmail aips2-newlist"
\end{verbatim}
\item A daily cron job run by root appends the tarzan.aliases-tail file to
tarzan's alias list.
\item Send a test message.  Put a symbolic link to the list recipients.
On tarzan, cd /export/aips++/Mail/aips2-newlist, ln -s ../../master/etc/aips2-newlist list
\end{enumerate}
More details on \aips mailing lists may be found in the
\htmlref{\aips System Manual} {email exploders}.
% LocalWords:  init dsed llnl unix longtables Refman htex helpfiles glish htdig
% LocalWords:  sdsu cron rundig conf htsearch httpd tarzan nrao wwwgnats cern
% LocalWords:  cygnus wyoung Cornwell Kemball contrib Lovell atnf
% LocalWords:  bima nfra config mailto sendmailto NCSA admin
% LocalWords:  helpdesk mailinglist newlist aipsmail frameonline homepage
% LocalWords:  betasearch homepages reportbug whatsnew Bridle's HTML html perl
% LocalWords:  texinfo docsys docs
